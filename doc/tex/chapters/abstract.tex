\begin{abstract}
%\addcontentsline{toc}{chapter}{Abstract}
This document contains the documentation of Dresden OCL, meaning, how to use and
how to extend tools provided by Dresden OCL. In the first part comprises the
general use of Dresden OCL and demonstrates use cases like OCL interpretation
and code generation. Afterwards, the second part comprises the technical
documentation of Dresden OCL, like its architecture and the adaptation of
further types of models and model instances to Dresden OCL.

Please be aware that Dresden OCL is a project developed at the Technische
Universit�t Dresden, Software Technology Group. Parts of the project have been 
designed and implemented during student theses and have been developed as
prototypes only. Thus, Dresden OCL is far from being complete. To report bugs 
and errors or request additional features or answers to specific questions visit
Dresden OCL's website~\cite{WWW:toolkit} or the project site at
\keyword{Github}~\cite{WWW:toolkitGithub}.

The procedures described in this manual were run and tested with
\keyword{\eclipseversion}~\cite{WWW:eclipse}. We recommend to use the
\keyword{Eclipse Modeling Tools Edition} which contains most required plug-ins
to run Dresden OCL. Otherwise you need to install at least the plug-ins enlisted
in Table~\ref{tab:software}. Alternatively, Dresden OCL may be used as a
stand-alone library for Java. If you want to use the stand-alone distribution,
you cannot use the \acs{GUI}s and editors provided with Dresden OCL since the
GUI elements depend on Eclipse. The use of the stand-alone distribution is
documented in Chapter~\ref{chapter:standalone}.

\end{abstract}